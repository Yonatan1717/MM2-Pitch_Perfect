\section{Teori}

\subsection{Fouriertransformasjon}

\subsubsection{Fourierrekker}
En Fourierrekke brukes til å beskrive periodiske funksjoner
— altså signaler som gjentar seg med en fast periode T
Ideen er at selv kompliserte periodiske signaler kan skrives 
som en sum av enkle harmoniske svingninger 
(sinus- og cosinusbølger). 

\[
f(t) = a_0 + \sum_{n=1}^{\infty} \left[ a_n \cos(\omega_n t) + b_n \sin(\omega_n t) \right]
\]
Her er 
\[
\omega_n = \frac{2\pi n}{T}
\]
vinkelfrekvensen til den $n$-te harmoniske, der $T$ er periodetiden. 
Koeffisientene beregnes som:

\[
a_n = \frac{2}{T} \int_{0}^{T} f(t)\cos(\omega_n t)\,dt, \qquad
b_n = \frac{2}{T} \int_{0}^{T} f(t)\sin(\omega_n t)\,dt
\]



\subsection{Diskre fouriertransformasjon (DFT)}

\subsubsection{Fast Fourier Transform (FFT)}
Den \textit{Fast Fourier Transform} (FFT) er en algoritme som effektivt beregner den diskrete Fourier-transformasjonen (DFT) og dens inverse (IDFT). 
Mens en direkte beregning av DFT krever et stort antall operasjoner med beregningskompleksitet $O(N^2)$, reduserer FFT dette til $O(N \log N)$ ved å faktorisere DFT-matrisen og utnytte dens underliggende strukturelle symmetrier. 
Her betegner $N$ antall datapunkter i signalet. 
Denne forbedringen gjør FFT til en av de mest betydningsfulle algoritmene innen moderne signalbehandling, og muliggjør rask og presis analyse av frekvensinnhold i digitale signaler.

FFT utnytter de innebygde symmetrieegenskapene og periodisiteten i de komplekse eksponentene som inngår i Fourier-transformasjonen. 
Ved å eliminere redundante beregninger oppnås en dramatisk reduksjon i prosesseringstid, noe som gjør sanntidsbehandling av lydopptak og andre signaler praktisk gjennomførbart. 
Som følge av dette har FFT fått en sentral rolle i et bredt spekter av anvendelser, inkludert lyd- og bildebehandling, ingeniørarbeid, vitenskapelige beregninger og matematisk analyse. 
Algoritmen omtales ofte som en av de viktigste numeriske metodene utviklet i moderne tid.

\paragraph{Cooley-Tukey-algoritmen.} 
Den mest utbredte implementasjonen av FFT er Cooley-Tukey-algoritmen, introdusert i 1965. 
Algoritmen følger et \textit{divide and conquer}-prinsipp, der signalet $x[n]$ deles i to delsekvenser: én bestående av partallsindekserte elementer og én av oddetallsindekserte elementer. 
DFT beregnes deretter separat for hver delsekvens, før resultatene kombineres ved å utnytte eksponentialfaktorene $e^{-2\pi i kn/N}$. 
Denne prosessen gjentas rekursivt til man står igjen med DFT-er av lengde 2, som er trivielle å beregne. 
På denne måten oppnås en betydelig reduksjon i beregningsmengde uten tap av nøyaktighet.




\subsection{Short-Time Fourier Transform (STFT)}

\subsection{Hva er lyd?}

\subsubsection{Mekanisk lyd}
I fysikken beskrives lyd som en mekanisk bølge som overføres gjennom et medium, for eksempel luft, vann eller faste stoffer. 
Lyd oppstår når en kilde skaper vibrasjoner som forplanter seg gjennom mediet i form av trykkvariasjoner. 
Som mekanisk fenomen kan lyd derfor forstås som oscillasjoner i trykk, stress, partikkelforskyvning og partikkelhastighet som funksjon av tid. 
Når disse trykkvariasjonene når trommehinnen i øret, oppfatter vi dem som lyd.

\subsubsection{Matematisk lyd}
Trykkvariasjonene som utgjør lyd kan modelleres matematisk som et kontinuerlig tidsavhengig signal:
\[
x(t): \mathbb{R} \rightarrow \mathbb{R}
\]
der $x(t)$ representerer lydens amplitude, altså trykkavviket fra likevekt, som funksjon av tiden $t$. 
I signalteori kan et slikt signal uttrykkes som en sum av sinusfunksjoner med ulike frekvenser, amplituder og faser ved hjelp av Fourier-transformasjonen. 
Ettersom de fleste lydsignaler består av en kombinasjon av flere frekvenskomponenter, kan Fourier-transformen brukes til å dekomponere signalet og identifisere dets frekvensinnhold. 
Dette danner grunnlaget for videre analyse og gjenkjenning av lyd og musikalske toner.

\subsection{Lyd som signal}
