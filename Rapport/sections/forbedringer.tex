\section{Forbedringer}
Selv om \textit{Pitch Perfect} fungerer tilfredsstillende for sanntids toneidentifikasjon, er det flere områder hvor programmet kan forbedres for å øke nøyaktigheten, brukervennligheten og robustheten.

En potensiell forbedring er å implementere mer avanserte støygate-algoritmer, som for eksempel spektral gating eller maskinlæringsbaserte metoder. Disse metodene kan bedre skille mellom ønsket signal og bakgrunnsstøy, spesielt i utfordrende lydmiljøer. I tillegg kan bruk av adaptiv vindusfunksjon, som justerer vindusstørrelsen basert på signalets egenskaper, bidra til å optimalisere tids- og frekvensoppløsningen dynamisk. Videre kan frekvensestimeringen forbedres ved å implementere mer sofistikerte interpolasjonsmetoder.

En annen viktig forbedring er å utvikle et grafisk brukergrensesnitt (GUI) som gjør det enklere for brukere å interagere med programmet, justere innstillinger og visualisere resultater i sanntid. Dette kan inkludere funksjoner som sanntids spektrogramvisning, tonehistogrammer og muligheten til å lagre analyseresultater for senere gjennomgang. I tillegg kan programmet gjøres mer effektivt med hensyn til tiden det tar å plotte grafer, spesielt ved plotting av spektrogram, for en bedre brukervennlig opplevelse.

En viktig forbedring (rapport messig) vil være å illustrerer effekten av RMS-gating, parabolisk toppinterpolasjon og vindusvalg ved å inkludere resultater med og uten denne funksjonen aktivert, for å tydelig vise hvordan den påvirker frekvensidentifikasjonen under varierende støyforhold.