\section{Innledning}
Lydanalyse er et viktig område innen moderne signalbehandling. Målet er å kunne beskrive, forstå og behandle signaler som varierer over tid, slik som lyd. Et lydsignal kan beskrives som en funksjon av tid, men mange av de mest interessante egenskapene – som tonehøyde, klang og overtoner – kommer tydeligere frem når signalet analyseres i frekvensdomenet. For å gjøre dette brukes Fourier-transformasjonen, et matematisk verktøy som gjør det mulig å representere et tidsdomene-signal som en sum av sinusformede bølger med ulike frekvenser, styrker og faser.

I dette prosjektet undersøkes hvordan Fourier-transformasjonen og dens diskrete varianter kan brukes til praktisk lydanalyse. Den diskrete Fourier-transformasjonen (DFT) og den raske Fourier-transformasjonen (FFT) implementeres for å beregne og visualisere frekvensinnholdet i digitale lydsignaler. Gjennom dette får man en forståelse av hvordan matematiske metoder kan brukes til å trekke ut informasjon fra reelle signaler.

Videre introduseres Short-Time Fourier Transform (STFT), som gjør det mulig å analysere hvordan frekvensinnholdet i et signal endrer seg over tid. Denne metoden gir en mer detaljert fremstilling av komplekse signaler, som for eksempel tale og musikk, der både tonehøyde og intensitet varierer kontinuerlig.

Prosjektets mål er å utvikle et system for sanntidsanalyse og visualisering av lyd ved hjelp av Fourier-baserte metoder. Arbeidet kombinerer teoretisk forståelse av Fourier-transformasjonen med praktisk programmering i Python, og viser hvordan matematiske prinsipper kan brukes til å løse konkrete problemer innen lydteknologi og signalbehandling.