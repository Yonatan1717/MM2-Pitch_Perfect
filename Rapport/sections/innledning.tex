\section{Innledning}
Lydanalyse er et sentralt tema innen moderne signalbehandling, hvor målet er å beskrive, forstå og manipulere signaler som varierer over tid. Et lydsignal kan representeres som en funksjon av tid, men mange av signalets egenskaper – som tonehøyde, klang og overtonestruktur – kommer tydeligere frem når signalet analyseres i frekvensdomenet. Fourier-transformasjonen utgjør et av de mest fundamentale verktøyene for denne typen analyse, da den gjør det mulig å uttrykke et tidsdomene-signal som en superposisjon av sinusformede komponenter med ulike frekvenser, amplituder og faser.

I dette prosjektet undersøkes hvordan Fourier-transformasjonen og dens diskrete varianter kan benyttes til praktisk lydanalyse. Spesielt fokuseres det på implementasjon av den diskrete Fourier-transformasjonen (DFT) og den raske Fourier-transformasjonen (FFT) for å identifisere og visualisere frekvensinnholdet i digitale lydsignaler. Gjennom dette oppnås en forståelse for hvordan matematiske prinsipper anvendes i digitale beregninger, og hvordan disse transformasjonene kan brukes for å trekke ut informasjon fra reelle signaler.

Videre introduseres Short-Time Fourier Transform (STFT) som en metode for tids-frekvensanalyse. STFT muliggjør dynamisk observasjon av frekvenskomponenter over tid, og gir dermed en mer komplett beskrivelse av lydsignalets utvikling enn tradisjonell frekvensanalyse alene. Denne metoden er spesielt relevant for lyd som varierer i både tonehøyde og intensitet, slik som musikk og tale.

Prosjektets hensikt er å utvikle et rammeverk for sanntidsanalyse og visualisering av lydsignaler ved bruk av Fourier-baserte metoder. Arbeidet kombinerer teoretisk forståelse av transformasjonsprinsipper med praktisk implementasjon i Python, og danner dermed et bindeledd mellom matematisk teori og anvendt signalbehandling. Resultatene skal gi innsikt i hvordan Fourier-transformasjonen kan brukes som et kraftig verktøy for å avdekke strukturen i komplekse, tidsvarierende lydsignaler.