\section{Sammendrag}

Dette prosjektet presenterer utviklingen av et sanntids lydanalysesystem basert på Fourier-transformasjonen. 
Målet var å undersøke hvordan teoretiske prinsipper fra signalbehandling kan brukes til å identifisere og visualisere frekvensinnholdet i lydsignaler. 
Systemet, kalt \textit{Pitch Perfect}, ble implementert i Python ved bruk av FFT og STFT for å analysere hvordan lydspektret endres over tid. 
Gjennom vindusfunksjoner, RMS-beregning og parabolsk toppinterpolasjon ble nøyaktigheten i frekvensestimatene forbedret. 
Resultatene viser at metoden gir presis toneidentifikasjon i sanntid, med små avvik som hovedsakelig skyldes støy og utstyrstoleranser. 
Prosjektet viser at matematiske metoder for signalanalyse kan implementeres effektivt i praktiske sanntidsapplikasjoner.
