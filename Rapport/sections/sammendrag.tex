\section*{Sammendrag}

I dette prosjektet ble det utviklet et Python-basert system for sanntidsanalyse av lydsignaler ved hjelp av Fourier-transformasjonen
 (DFT, FFT og STFT). Målet var å identifisere og visualisere frekvensinnhold i et tidsdomene-signal. Systemet benytter en 
 produsent–konsument-arkitektur og parabolsk toppinterpolasjon for økt presisjon. Ved en samplingsfrekvens på 48~kHz og et 
 FFT-vindu på 2048 prøver oppnås en frekvensoppløsning på 23,44~Hz, med typisk toneavvik under 1~Hz og forsinkelse under 30~ms. 
 Vindusfunksjoner som Hann og Hamming reduserer spektral lekkasje. Resultatene viser at systemet gir stabil og nøyaktig toneidentifikasjon 
 i sanntid, og illustrerer hvordan teoretiske egenskaper ved Fourier-transformasjonen kan anvendes i praksis.
