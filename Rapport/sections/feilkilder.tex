\section{Feilkilder}
Flere forhold kan ha påvirket nøyaktigheten og stabiliteten i resultatene våre. En viktig feilkilde er bakgrunnsstøy og variasjoner i opptaksmiljøet. Selv om programmet har en adaptiv støygate, vil endringer i bakgrunnsstøy, for eksempel tastetrykk, pust eller romklang, kunne føre til at terskelen enten blir for høy eller for lav. Dette kan igjen føre til at svake toner blir ignorert, eller at støy feiltolkes som en faktisk tone.

En annen feilkilde ligger i opptaksutstyret. Mikrofonens kvalitet og plassering påvirker både lydstyrke og frekvensrespons, og enkle datamikrofoner kan forvrenge signalet eller ha begrenset båndbredde. Dette påvirker resultatet av Fourier-analysen, spesielt i de høyere frekvensene.