\section{Metode}
\subsection{Sanntids innhenting av lyd}
Lyd hentes kontinuerlig inn fra mikrofonen ved hjelp av biblioteket PyAudio, som muliggjør direkte tilgang til lydstrømmer i sanntid. Signalet samles inn som 16-biters heltallsprøver (int16) og deles opp i mindre blokker, eller såkalte chunks, på 1024 prøver. Disse blokkene legges i en kø (Queue), som fungerer som et bindeledd mellom to tråder: én som produserer lyddata, og én som forbruker og analyserer dem.

Denne produsent–konsument-strukturen gjør at innsamlingen av lyd og selve analysen kan foregå parallelt uten å forstyrre hverandre. Hvis køen blir full, fjernes de eldste dataene automatisk for å unngå forsinkelser og sikre jevn oppdatering.

\subsection{Fourier-analyse av lydsignalet}
For å analysere signalet brukes Fast Fourier Transform (FFT), som regnes ut med funksjoner fra NumPy-biblioteket. FFT omgjør lydsignalet fra tidsdomenet til frekvensdomenet, slik at vi kan se hvilke frekvenser som er mest dominerende i hvert lydvindu.

\subsection{Støygating og glatting}
For å unngå at bakgrunnsstøy og svake signaler forstyrrer resultatene, beregnes lydnivået (RMS) i hvert vindu. Det brukes en adaptiv støymodell som justerer seg over tid etter hvor høyt bakgrunnsnivået er. Dersom lydnivået i et vindu er lavere enn en bestemt terskel, blir vinduet hoppet over.

\subsection{Beregning av tone og note}
Etter at de sterkeste frekvensene er funnet, brukes en funksjon som oversetter frekvensene til musikalske noter. Dette gjøres ved hjelp av den matematiske sammenhengen mellom frekvens og note i den tempererte skalaen, der A4 defineres som 440 Hz. Frekvensen omgjøres til en tilsvarende MIDI-verdi, som deretter rundes av til nærmeste hele notenummer. Programmet beregner også avviket i cent (en hundredel av et halvt tonetrinn) og viser hvor mye den faktiske frekvensen avviker fra den ideelle noten.

For å forbedre presisjonen brukes kvadratisk interpolasjon rundt toppene i frekvensspekteret. Denne metoden finner et mer nøyaktig toppunkt mellom FFT-binene, slik at frekvensen kan bestemmes med høyere oppløsning enn FFT-en i seg selv tillater.