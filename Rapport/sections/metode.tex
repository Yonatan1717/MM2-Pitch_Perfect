\section{Metode}
\subsection{Sanntids innhenting av lyd}
Lyd hentes kontinuerlig inn fra mikrofonen ved hjelp av biblioteket PyAudio, som muliggjør direkte tilgang til lydstrømmer i sanntid. Signalet samles inn som 16-biters heltallsprøver (int16) og deles opp i mindre blokker, eller såkalte chunks, på 1024 prøver. Disse blokkene legges i en kø (Queue), som fungerer som et bindeledd mellom to tråder: én som produserer lyddata, og én som forbruker og analyserer dem.

Denne produsent–konsument-strukturen gjør at innsamlingen av lyd og selve analysen kan foregå parallelt uten å forstyrre hverandre. Hvis køen blir full, fjernes de eldste dataene automatisk for å unngå forsinkelser og sikre jevn oppdatering.

\subsection{Fourier-analyse av lydsignalet}
For å analysere signalet brukes Fast Fourier Transform (FFT), som regnes ut med funksjoner fra NumPy-biblioteket. FFT omgjør lydsignalet fra tidsdomenet til frekvensdomenet, slik at vi kan se hvilke frekvenser som er mest dominerende i hvert lydvindu.

\subsection{Støygating og glatting}

\subsection{Beregning av tone og note}