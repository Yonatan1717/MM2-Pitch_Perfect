\section{Konklusjon}

Arbeidet har demonstrert hvordan Fourier-transformasjonen og dens diskrete varianter kan brukes til effektiv sanntidsanalyse av lydsignaler. 
Gjennom implementering av STFT i Python ble både tids- og frekvensinnhold visualisert på en måte som stemmer godt overens med teoretiske forventninger. 
Bruken av Hanning-vindu reduserte spectral leakage, og parabolsk toppinterpolasjon ga mer presise frekvensestimater. 
Resultatene bekrefter at systemet nøyaktig kan identifisere dominerende toner, som vist ved analyse av en A4-tone på 440~Hz. 

Små avvik mellom beregnet og observert frekvens skyldes hovedsakelig støy, begrenset oppløsning og variasjoner i opptaksutstyr, men påvirker ikke hovedkonklusjonen. 
Prosjektet viser at sanntids frekvensanalyse kan realiseres med høy nøyaktighet ved hjelp av relativt enkle metoder, og danner et godt grunnlag for videre utvikling av mer avanserte lydanalyseverktøy.
