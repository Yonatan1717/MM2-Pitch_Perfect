\section{Resultater}
I dette kapittelet presenterer vi resultater fra \textit{Pitch Perfect} alt av test data finnes er vedlagt i vedleg \hyperlink{vedlegg}{\textcolor{blue}{1}}. vi vil vise følgende analyser:\\[1em]
\textbf{Data over lengre tid (5 sekunder):}
\begin{itemize}
    \item Tidsdomene signal
    \item Frekvensdomene signal
    \item Spektrogram
\end{itemize}
\textbf{Toneidentifikasjon i sanntid (siste ramme):}
\begin{itemize}
    \item Tidsdomene signal
    \item Frekvensdomene signal
\end{itemize}

\subsection{Data over lengre tid (5 sekunder)}
\subsubsection{Tidsdomene signal}
tidsdomene signalet viser hvordan lydtrykket varierer over tid. i figur \ref{fig:tidssignal} vises tidsdomene signal generert av \textit{Pitch Perfect} for signalet. Her ser vi variasjonene i lydtrykk over tid, som representerer lydens amplitude og frekvensinnhold. tidsdomene signalet gir oss en direkte visualisering av lydens bølgeform, og kan brukes til å identifisere ulike egenskaper ved lydsignalet, som for eksempel periodisitet og transienter. 
\begin{figure}[H]
    \centering
    \includegraphics[width=0.8\textwidth]{media/time_domain_(non-windowed,_5s).png}
    \caption{tidsdomene signal generert av \textit{Pitch Perfect}.}
    \label{fig:tidssignal}
\end{figure}
til sammenligning vises tidsdomene signalet fra kilden i figur \ref{fig:tidssignal_ref}.
\begin{figure}[H]
    \centering
    \includegraphics[width=0.8\textwidth]{media/time_domain_(non-windowed,_5s)_kilde.png}
    \caption{tidsdomene signal fra kilden.}
    \label{fig:tidssignal_ref}
\end{figure}
\subsubsection{Frekvensdomene signal}
frekvensdomene signalet viser hvordan lydens energi er fordelt over forskjellige frekvenser. i figur \ref{fig:frekvenssignal} vises frekvensdomene signal generert av \textit{Pitch Perfect} for signalet. her kan vi se hvilke frekvenser som er mest fremtredende i lydsignalet, og deres respektive amplituder. frekvensdomene signalet gir oss innsikt i lydens spektrale innhold, og kan brukes til å identifisere toner, harmoniske strukturer og andre frekvensrelaterte egenskaper ved lydsignalet. 
\begin{figure}[H]
    \centering
    \includegraphics[width=0.8\textwidth]{media/fft_(windowed,_5s).png}
    \caption{frekvensdomene signal generert av \textit{Pitch Perfect}.}
    \label{fig:frekvenssignal}
\end{figure}

\subsubsection{Spektrogram}
et spektrogram er en visuell representasjon av frekvensinnholdet i et lydsignal over tid. det viser hvordan amplituden av forskjellige frekvenser varierer i løpet av lydsignalets varighet. dette gir oss et 2d-bilde der x-aksen representerer tid, y-aksen representerer frekvens, og fargen eller intensiteten til hvert punkt representerer amplituden til den aktuelle frekvensen på det aktuelle tidspunktet. i figur \ref{fig:spektrogram} vises et eksempel på et spektrogram generert av \textit{Pitch Perfect}. her kan vi se hvordan forskjellige frekvenser dukker opp og forsvinner over tid, noe som gir oss innsikt i lydsignalets struktur og innhold.
\begin{figure}[H]
    \centering
    \includegraphics[width=0.8\textwidth]{media/spectrogram.png}
    \caption{spektrogram generert av \textit{Pitch Perfect}.}
    \label{fig:spektrogram}
\end{figure}
Platformen signalet ble hentet fra gir følgende spektrogram for sammenligning i figur \ref{fig:spektrogram_ref}.
\begin{figure}[H]
    \centering
    \includegraphics[width=0.8\textwidth]{media/spectrogram_kilde.png}
    \caption{spektrogram fra kilden.}
    \label{fig:spektrogram_ref}
\end{figure}
\clearpage
\subsection{Tone identifikasjon i sanntid (siste ramme)}
\subsubsection{Tidsdomene signal}
I figur \ref{fig:tidssignal_a4} vises tidsdomene signal generert av \textit{Pitch Perfect} for den siste rammen av signalet. her kan vi se variasjonene i lydtrykk over tid for den spesifikke rammen, som representerer lydens amplitude og frekvensinnhold i dette øyeblikket. Tidsdomenesignalet gir oss en direkte visualisering av lydens bølgeform i den siste rammen, og kan brukes til å identifisere ulike egenskaper ved lydsignalet, som for eksempel periodisitet.
\begin{figure}[H]
    \centering
    \includegraphics[width=1\textwidth]{Media/Time_Domain_(Non-Windowed).png}
    \caption{tidsdomene signal generert av \textit{Pitch Perfect} for den siste rammen av signalet.}
    \label{fig:tidssignal_a4}
\end{figure}
Ut ifra figuren kan vi se at signalet har en periodisitet som stemmer overens med frekvensen til en A4 note (440 Hz), dette har vi funnet ved å måle avstanden mellom to påfølgende topper i signalet:
\[
    T =  top_2 - top_1 \approx 0.00475 - 0.00248 = 0.00227 \text{ sek}
\]
\[
    f = \frac{1}{T} = \frac{1}{0.00227} \approx 440.53 \text{ Hz}
\]
\clearpage
\noindent
Som nevnt blir hver ramme ganget med et Hann-vindu, og  figur \ref{fig:tidssignal_a4_vinduet} viser tidsdomene signalet for den siste rammen etter påføring av vinduet.
\begin{figure}[H]
    \centering
    \includegraphics[width=.9\textwidth]{Media/Time_Domain_(Windowed).png}
    \caption{tidsdomene signal generert av \textit{Pitch Perfect} for den siste rammen av signalet etter påføring av Hann-vindu.}
    \label{fig:tidssignal_a4_vinduet}
\end{figure}
\subsubsection{Frekvensdomene signal}
I figur \ref{fig:frekvenssignal_a4} vises frekvensdomene signal generert av \textit{Pitch Perfect} for den siste rammen av signalet.
\begin{figure}[H]
    \centering
    \includegraphics[width=1\textwidth]{Media/Frequency_Spectrum.png}
    \caption{frekvensdomene signal generert av \textit{Pitch Perfect} for den siste rammen av signalet.}
    \label{fig:frekvenssignal_a4}
\end{figure}
Ut ifra figuren kan vi se at den dominerende frekvensen i signalet er rundt 440 Hz, som stemmer overens med frekvensen til en A4 note (440 Hz). Dette bekrefter at signalet inneholder en sterk komponent ved denne frekvensen, noe som er forventet gitt at vi analyserer en ren A4 note. Denne informasjonen ser vi også i tabellen over de sterkeste frekvensene i figur \ref{fig:Top_10}.
\begin{figure}[H]
    \centering
    \includegraphics[width=.5\textwidth]{Media/stats.png}
    \caption{Sterkeste frekvens identifisert av \textit{Pitch Perfect} for den siste rammen av signalet.}
    \label{fig:Top_10}
\end{figure}
