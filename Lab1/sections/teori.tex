\section{Teori}

\subsection{Hva er lyd?}

\subsubsection{Mekanisk lyd}
I fysikken beskrives lyd som en mekanisk bølge som overføres gjennom et medium, for eksempel luft, vann eller faste stoffer. 
Lyd oppstår når en kilde skaper vibrasjoner som forplanter seg gjennom mediet i form av trykkvariasjoner. 
Som mekanisk fenomen kan lyd derfor forstås som oscillasjoner i trykk, stress, partikkelforskyvning og partikkelhastighet som funksjon av tid. 
Når disse trykkvariasjonene når trommehinnen i øret, oppfatter vi dem som lyd.

\subsubsection{Matematisk lyd}
Trykkvariasjonene som utgjør lyd kan modelleres matematisk som et kontinuerlig tidsavhengig signal:
\[
x(t): \mathbb{R} \rightarrow \mathbb{R}
\]
der $x(t)$ representerer lydens amplitude, altså trykkavviket fra likevekt, som funksjon av tiden $t$. 
I signalteori kan et slikt signal uttrykkes som en sum av sinusfunksjoner med ulike frekvenser, amplituder og faser ved hjelp av Fourier-transformasjonen. 
Ettersom de fleste lydsignaler består av en kombinasjon av flere frekvenskomponenter, kan Fourier-transformen brukes til å dekomponere signalet og identifisere dets frekvensinnhold. 
Dette danner grunnlaget for videre analyse og gjenkjenning av lyd og musikalske toner.

\subsection{Lyd som signal}
