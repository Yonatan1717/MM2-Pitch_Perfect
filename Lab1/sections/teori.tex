\section{Teori}

\subsection{Fouriertransformasjon}

\subsubsection{Fourierrekker}
En Fourierrekke brukes til å beskrive periodiske funksjoner
— altså signaler som gjentar seg med en fast periode T
Ideen er at selv kompliserte periodiske signaler kan skrives 
som en sum av enkle harmoniske svingninger 
(sinus- og cosinusbølger). 

\[
f(t) = a_0 + \sum_{n=1}^{\infty} \left[ a_n \cos(\omega_n t) + b_n \sin(\omega_n t) \right]
\]
Her er 
\[
\omega_n = \frac{2\pi n}{T}
\]
vinkelfrekvensen til den $n$-te harmoniske, der $T$ er periodetiden. 
Koeffisientene beregnes som:

\[
a_n = \frac{2}{T} \int_{0}^{T} f(t)\cos(\omega_n t)\,dt, \qquad
b_n = \frac{2}{T} \int_{0}^{T} f(t)\sin(\omega_n t)\,dt
\]



\subsection{Diskre fouriertransformasjon (DFT)}

\subsubsection{Fast Fourier Transform}

\subsection{Short-Time Fourier Transform (STFT)}

